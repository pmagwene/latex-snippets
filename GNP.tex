\documentclass[11pt]{article}

\usepackage{MinionPro}
\usepackage{MnSymbol}

\makeatletter
\newcommand{\superimpose}[2]{%
  {\ooalign{$#1\@firstoftwo#2$\cr\hfil$#1\@secondoftwo#2$\hfil\cr}}}
\makeatother
\newcommand{\GE}{\mathpalette\superimpose{{G}{E}}}

\newcommand\opn{\mathrel{\ooalign{{G}\cr
  \hidewidth\raise.1ex{E}}}}
  
%\newcommand{\GNP}{{\ooalign{\raisebox{1ex}{G}\cr
%  \hidewidth\raisebox{-1ex}{E}}}}

%\newcommand{\GNP}{\ensuremath{\ooalign{\raisebox{1ex}{$G$}\cr
%  \hidewidth\raisebox{-1ex}{$E$}}}}

% see
% http://tex.stackexchange.com/questions/22371/subseteq-circ-as-a-single-symbol-open-subset/22375
% for an explanation of ooalign

\newcommand{\GP}{\ensuremath{\ooalign{\raisebox{1ex}{$G$}\cr\hidewidth\raisebox{-1ex}{$E$}}}
{\ooalign{\raisebox{0.85ex}{${\scriptstyle\searrow}$}\cr\hidewidth\raisebox{-0.5ex}{${\mkern1mu\scriptstyle\nearrow}$}  }  }}


\newcommand{\GTP}{\ensuremath{\ooalign{\raisebox{1ex}{$G$}\cr\hidewidth\raisebox{-1ex}{$E$}}}\!\ensuremath{{{\scriptstyle\searrow}\atop{\scriptstyle\nearrow}}}}

\newcommand{\GoverE}{\ensuremath{\ooalign{\raisebox{0.9ex}{$G$}\cr\hidewidth\raisebox{-0.9ex}{$E$}}}}

\newcommand{\OUarrow}{\ensuremath{
\ooalign{%
\raisebox{0.75ex}{${\scriptstyle\searrow}$}\cr\raisebox{-0.65ex}{${\scriptstyle\nearrow}$}%
}%
}%
}

\newcommand{\GENP}{\GoverE\OUarrow\ensuremath{N\!\to\!P}}

\newcommand{\NOE}{\ensuremath{\ooalign{\hfil$N$\hfil\cr\hfil\raisebox{-2ex}{$\uparrow$}\hfil\cr\hfil\raisebox{-4.1ex}{$E$}\cr}}}

\newcommand{\EON}{\ensuremath{\ooalign{\hfil$N$\hfil\cr\hfil\raisebox{2ex}{$\downarrow$}\hfil\cr\hfil\raisebox{4.1ex}{$E$}\cr}}}


\begin{document}

\section*{Using various TeX/LaTeX macros}

\GENP\ is easy.

\[
{G}\to\NOE\to{P}
\]
That looks good, doesn't it! The E over N version seems to require a medskip:
\medskip

\[
{G}\to\EON\to{P}
\]


\section*{Using substack}

The genotype-network-phenotype mapping is:    
\[
\substack{G\rightarrow\\E\rightarrow}N \rightarrow P
\]    
And here it is inline -- $\substack{G\rightarrow\\E\rightarrow}N \rightarrow P$. And now inline with \verb|\textstyle| G and E -- $\substack{{\textstyle G}\searrow\\ {\textstyle E}\nearrow}N \to P$.
This also uses substack, but with N as a \verb|\mathop{}|: $G \rightarrow \mathop{N}_{\substack{\uparrow\\E}} \rightarrow P$? 

    
\section*{Using genfrac}    
Here's another way to represent it, using the \verb|genfrac| command from \verb|amsmath|.
\[
G \rightarrow \mathop{N}_{\genfrac{}{}{0pt}{0}{\uparrow}{E}} \rightarrow P
\]
This is also using \verb|genfrac| but with the E on top:
\[
G \rightarrow \mathop{N}^{\genfrac{}{}{0pt}{0}{E}{\downarrow}} \rightarrow P
\]
The problem with genfrac is that it doesn't produce a very nice looking inline version -- \(G \rightarrow \mathop{N}_{\genfrac{}{}{0pt}{0}{\uparrow}{E}} \rightarrow P\)?

\section*{Using atop}

Display style atop:
\[
{G \atop E} \to N \to P
\]
And here it is inline -- \( {G \atop E} \to N \to P \). This also uses \verb|atop| but uses a double arrow symbol from the MnSymbol package -- \( {G \atop E} \rightrightarrows N \to P \). And now with squiggly arrows and \verb|\textstyle| G and E.  -- \( {{\textstyle G} \atop {\textstyle E}} {\selsquigarrow \atop \nersquigarrow} N \to P \)

This is \verb|atop| as well, but using script style arrows:
\[
{G \atop E} {{\scriptstyle \searrow} \atop {\scriptstyle \nearrow}} N \to P
\]
Here's another way to represent it using atop:
\[
    G \rightarrow \mathop{N}_{\displaystyle {\uparrow \atop E}} \rightarrow P
\]

\section*{Using atop with sideset}
Here's some more takes using \verb|\atop| in combination with \verb|\sideset{}|:
\[
{\sideset{}{_{\searrow}}{\mathop{G}} \atop \sideset{}{^{\nearrow}}{\mathop{E}}}  N \to P
\]
And the inline version -- \(
{\sideset{}{_{\searrow}}{\mathop{G}} \atop \sideset{}{^{\nearrow}}{\mathop{E}}}\!N\!\to\!P
\). Followed by more text, how does it effect the interline spacing?  Does it make things look funny?  Well, does it?


\[
{{\textstyle G} \atop {\textstyle E}} {\searrow \atop \nearrow} N \to P
\]
And inline it looks like: \(
{{\textstyle G} \atop {\textstyle E}} {\searrow \atop \nearrow} N \to P
\)

And the inline version -- \(
{\sideset{}{_{\searrow}}{\mathop{\textstyle G}} \atop \sideset{}{^{\nearrow}}{\mathop{\textstyle E}}}\!N\!\to\!P
\).

%$\sideset{}{_{\searrow}}{\substack{{\textstyle G} \\ {\textstyle E}}} N \to P$.


% Here's some rlap magic: \rlap{\raisebox{1ex}{G}}{\raisebox{-1ex}{E}}
% 
% \makebox[0pt][c]{\raisebox{1ex}{G}\hspace*{-0.5em}\raisebox{-1ex}{E}}
% 
% $\GE$

% \def\clap#1{\hbox to 0pt{\hss#1\hss}}
% 
% Here's some clap magic: \clap{\raisebox{1ex}{G}}{\raisebox{-1ex}{E}}

% \GNP $\to\!N\!\to\!P$
% 
% \GP
% 
% \GTP

\end{document}